\documentclass[aspectratio=169,11pt]{beamer}

% Theme
\usetheme{Madrid}
\usecolortheme{whale}
\setbeamertemplate{navigation symbols}{}
\setbeamertemplate{footline}[frame number]

% Packages
\usepackage{graphicx}
\usepackage{booktabs}
\usepackage{amsmath}
\usepackage{xcolor}
\usepackage{svg}  % For SVG support - requires inkscape

% Colors
\definecolor{celegans}{RGB}{76,153,0}
\definecolor{contact}{RGB}{204,102,0}
\definecolor{lineage}{RGB}{51,102,204}
\definecolor{context}{RGB}{128,0,128}

% Title
\title[NemaContext]{NemaContext: The Organism as Context}
\subtitle{Flow Matching Transformers for Digital Embryogenesis}
\author{Progress Report}
\date{\today}

\begin{document}

% ============================================================
% Title Slide
% ============================================================
\begin{frame}
\titlepage
\end{frame}

% ============================================================
% Outline
% ============================================================
\begin{frame}{Outline}
\tableofcontents
\end{frame}

% ============================================================
\section{Core Philosophy: The Organism as Context}
% ============================================================

\begin{frame}{The Central Thesis}
\begin{center}
\Large
\textit{``A cell is not an island.\\
Its identity, position, and fate are defined not by intrinsic properties alone,\\
but by its place within the developing whole.''}
\end{center}

\vspace{1.5em}

\begin{columns}
\column{0.48\textwidth}
\begin{block}{The Name: NemaContext}
\textbf{Nema}(tode) + \textbf{Context}

The organism \textit{is} the context that gives meaning to each cell.
\end{block}

\column{0.48\textwidth}
\begin{block}{The Computational Principle}
\[
\text{Cell}_i = f(\text{Cell}_i, \textcolor{context}{\text{All Other Cells}})
\]
Each cell's representation is computed as a function of the \textbf{entire embryo}.
\end{block}
\end{columns}
\end{frame}

\begin{frame}{Three Levels of Developmental Context}
\begin{center}
% Figure: Three context levels diagram
% TODO: Create figures/context_levels.svg
\fbox{\parbox{0.8\textwidth}{\centering\vspace{3cm}\textbf{[Figure: context\_levels.svg]}\\\small Lineage Context + Molecular Context + Spatial Context\vspace{3cm}}}
\end{center}

\begin{columns}
\column{0.32\textwidth}
\begin{block}{1. Lineage Context}
\small
\textbf{Temporal}: Where did this cell come from?

The Sulston lineage = developmental memory
\end{block}

\column{0.32\textwidth}
\begin{block}{2. Molecular Context}
\small
\textbf{State}: What genes is this cell expressing?

Transcriptome = current functional identity
\end{block}

\column{0.32\textwidth}
\begin{block}{3. Spatial Context}
\small
\textbf{Relational}: Who are this cell's neighbors?

Contact graph = signaling potential
\end{block}
\end{columns}

\vspace{0.5em}
\centering
\textcolor{red}{\textbf{The problem: Spatial context is UNKNOWN for late embryo (380--830 min)}}
\end{frame}

\begin{frame}{Why \textit{C. elegans}?}
\begin{columns}
\column{0.55\textwidth}
\begin{block}{The Only Tractable System}
\begin{itemize}
    \item \textbf{Invariant lineage}: 100\% deterministic divisions
    \item \textbf{Complete connectome}: Adult wiring known
    \item \textbf{Lineage-resolved transcriptomics}: 234K cells
    \item \textbf{4D morphological atlas}: CShaper, WormGUIDES
    \item \textbf{Small cell count}: 959 terminal cells
\end{itemize}
\end{block}

\begin{block}{Digital Embryogenesis is Feasible}
\textit{C. elegans} is the \textbf{only} organism where we can attempt to generate complete embryo trajectories.
\end{block}

\column{0.42\textwidth}
% Figure: C. elegans lineage tree
% TODO: Create figures/lineage_tree.svg
\fbox{\parbox{\textwidth}{\centering\vspace{4cm}\textbf{[Figure: lineage\_tree.svg]}\\\small P0 → 959 cells\vspace{1cm}}}
\end{columns}
\end{frame}

% ============================================================
\section{The Temporal Coverage Gap}
% ============================================================

\begin{frame}{Data Landscape: Trimodal Integration}
\begin{center}
% Figure: Trimodal data integration
% TODO: Create figures/trimodal_integration.svg
\fbox{\parbox{0.9\textwidth}{\centering\vspace{3cm}\textbf{[Figure: trimodal\_integration.svg]}\\\small Transcriptome (234K) + Spatial (1.3K) + Morphology (1.2K) → Unified AnnData\vspace{2cm}}}
\end{center}

\begin{columns}
\column{0.32\textwidth}
\centering
\textbf{Transcriptome}\\
Large et al. 2025\\
234,888 cells\\
0--830 min

\column{0.32\textwidth}
\centering
\textbf{Spatial}\\
WormGUIDES\\
1,341 cells\\
20--380 min

\column{0.32\textwidth}
\centering
\textbf{Morphology}\\
CShaper\\
1,234 cells\\
20--380 min
\end{columns}
\end{frame}

\begin{frame}{The Temporal Coverage Gap}
\begin{center}
% Figure: Temporal coverage gap timeline
% TODO: Create figures/temporal_gap.svg
\fbox{\parbox{0.9\textwidth}{\centering\vspace{4cm}\textbf{[Figure: temporal\_gap.svg]}\\\small Timeline showing coverage gap: CShaper/WormGUIDES (20-380 min) vs Large2025 (0-830 min)\vspace{2cm}}}
\end{center}

\vspace{0.5em}
\textbf{Problem}: For cells in the 380--830 min window, we have:
\begin{itemize}
    \item[\checkmark] Full transcriptome data (gene expression)
    \item[\checkmark] Lineage identity (from Sulston tree)
    \item[$\times$] \textcolor{red}{No spatial coordinates}
    \item[$\times$] \textcolor{red}{No contact graph}
\end{itemize}

\vspace{0.5em}
\textbf{Our Question}: Can we \textbf{infer} spatial context from lineage + molecular context?
\end{frame}

\begin{frame}{The Contact Graph Problem}
\begin{block}{Why Contact Graphs Matter}
\begin{itemize}
    \item \textbf{Notch signaling}: Requires direct cell-cell contact (GLP-1/APX-1, LIN-12/LAG-2)
    \item \textbf{Inductive fate decisions}: Neighbors determine cell identity
    \item \textbf{Tissue organization}: Contacts define morphogenesis
\end{itemize}
\end{block}

\vspace{0.5em}

\begin{block}{The Inverse Problem}
\begin{center}
\Large
\textbf{Given}: Lineage + Transcriptome (what we know)\\[0.3em]
\textbf{Infer}: Contact Graph (what we need)
\end{center}
\end{block}

\vspace{0.5em}
\textbf{Hypothesis}: The organism provides sufficient context. If we know a cell's developmental history and current molecular state, we can predict its spatial relationships.
\end{frame}

% ============================================================
\section{Why Not GNNs? The Bitter Lesson}
% ============================================================

\begin{frame}{The GNN Approach (and Why We Reject It)}
\begin{columns}
\column{0.48\textwidth}
\begin{block}{Standard GNN Paradigm}
\begin{itemize}
    \item Assume graph structure is \textbf{given}
    \item Message passing along edges
    \item Learn node representations
\end{itemize}
\end{block}

\begin{alertblock}{Fundamental Problem}
GNNs require the graph as \textbf{input}.\\[0.3em]
But the contact graph is exactly what we're trying to \textbf{predict}!\\[0.3em]
\textcolor{red}{Chicken-and-egg problem.}
\end{alertblock}

\column{0.48\textwidth}
% Figure: GNN chicken-and-egg problem
% TODO: Create figures/gnn_problem.svg
\fbox{\parbox{\textwidth}{\centering\vspace{4cm}\textbf{[Figure: gnn\_problem.svg]}\\\small GNN needs graph → Graph is unknown → ???\vspace{1cm}}}
\end{columns}
\end{frame}

\begin{frame}{The Bitter Lesson}
\begin{block}{Rich Sutton (2019)}
\textit{``The biggest lesson that can be read from 70 years of AI research is that \textbf{general methods that leverage computation} are ultimately the most effective, and by a large margin.''}
\end{block}

\vspace{0.5em}

\begin{columns}
\column{0.48\textwidth}
\begin{alertblock}{GNN Approach}
\begin{itemize}
    \item[\textbf{--}] Encodes \textbf{human knowledge} into architecture
    \item[\textbf{--}] Hand-crafted graph topology
    \item[\textbf{--}] Message passing = limited context
    \item[\textbf{--}] Over-smoothing with depth
    \item[\textbf{--}] Poor GPU utilization (sparse ops)
\end{itemize}
\end{alertblock}

\column{0.48\textwidth}
\begin{exampleblock}{Transformer Approach}
\begin{itemize}
    \item[\textbf{+}] \textbf{Learns from data}
    \item[\textbf{+}] No assumed topology
    \item[\textbf{+}] Full attention = organism as context
    \item[\textbf{+}] Scales with depth
    \item[\textbf{+}] Excellent GPU utilization (dense ops)
\end{itemize}
\end{exampleblock}
\end{columns}
\end{frame}

\begin{frame}{Transformers Embody ``Organism as Context''}
\begin{center}
% Figure: Transformer attention as organism context
% TODO: Create figures/transformer_context.svg
\fbox{\parbox{0.85\textwidth}{\centering\vspace{4cm}\textbf{[Figure: transformer\_context.svg]}\\\small Self-attention: Each cell attends to ALL other cells in embryo\vspace{2cm}}}
\end{center}

\begin{block}{The Mathematical Formalization}
\[
\text{Cell}_i^{\text{repr}} = \sum_{j \in \text{Embryo}} \text{Attention}(Q_i, K_j) \cdot V_j
\]
\textbf{Every cell's representation is computed from the entire organism.}\\
This IS ``organism as context'' --- made computational.
\end{block}
\end{frame}

% ============================================================
\section{Flow Matching Transformers}
% ============================================================

\begin{frame}{Architecture Overview}
\begin{center}
% Figure: Flow Matching Transformer architecture
% TODO: Create figures/architecture.svg
\fbox{\parbox{0.9\textwidth}{\centering\vspace{5cm}\textbf{[Figure: architecture.svg]}\\\small Cell Tokens → Pairwise Transformer → Flow Matching Head → Contact Graph\vspace{1.5cm}}}
\end{center}

\begin{columns}
\column{0.32\textwidth}
\centering
\textbf{Input}\\
Cell tokens\\
(Transcriptome + Lineage + Time)

\column{0.32\textwidth}
\centering
\textbf{Encoder}\\
Pairwise Transformer\\
(Axial Attention)

\column{0.32\textwidth}
\centering
\textbf{Output}\\
Contact Graph\\
(Generated via Flow Matching)
\end{columns}
\end{frame}

\begin{frame}{Cell Tokenization}
\begin{block}{Each Cell = One Token}
\[
\text{Token}_i = [\underbrace{\mathbf{e}_i^{\text{expr}}}_{\text{scGPT}} \| \underbrace{\mathbf{e}_i^{\text{lin}}}_{\text{Binary Path}} \| \underbrace{\mathbf{e}_i^{\text{time}}}_{\text{Sinusoidal}} \| \underbrace{\mathbf{e}_i^{\text{morph}}}_{\text{CShaper}}]
\]
\end{block}

\begin{columns}
\column{0.48\textwidth}
\begin{block}{Transcriptome Embedding}
\begin{itemize}
    \item scGPT foundation model (768-dim)
    \item Or: PCA + MLP (lightweight)
    \item Captures molecular state
\end{itemize}
\end{block}

\begin{block}{Lineage Encoding}
\begin{itemize}
    \item Binary path from zygote
    \item Example: ``ABplp'' → [0,1,0,1,0,...]
    \item Encodes developmental history
\end{itemize}
\end{block}

\column{0.48\textwidth}
\begin{block}{Temporal Encoding}
\begin{itemize}
    \item Sinusoidal (Transformer-style)
    \item Developmental time: 0--830 min
    \item Captures temporal position
\end{itemize}
\end{block}

\begin{block}{Morphology (when available)}
\begin{itemize}
    \item Volume, surface area, sphericity
    \item From CShaper (early embryo)
    \item Imputed for late embryo
\end{itemize}
\end{block}
\end{columns}
\end{frame}

\begin{frame}{Flow Matching: Generative Graph Modeling}
\begin{columns}
\column{0.55\textwidth}
\begin{block}{Why Flow Matching?}
\begin{itemize}
    \item \textbf{Generative}: Produces graphs, not just scores
    \item \textbf{Deterministic sampling}: Faster than diffusion
    \item \textbf{Stable training}: No score matching issues
    \item \textbf{Structured outputs}: Natural for adjacency matrices
\end{itemize}
\end{block}

\begin{block}{The Formulation}
Transform noise $\mathbf{Z} \sim \mathcal{N}(0,1)$ to adjacency $\mathbf{A}$:
\[
\mathbf{Z} \xrightarrow{\text{Flow } \psi_t} \mathbf{A}
\]
Conditioned on: (transcriptome, lineage, time)
\end{block}

\column{0.42\textwidth}
% Figure: Flow matching illustration
% TODO: Create figures/flow_matching.svg
\fbox{\parbox{\textwidth}{\centering\vspace{4cm}\textbf{[Figure: flow\_matching.svg]}\\\small Noise → Flow → Contact Graph\vspace{1cm}}}
\end{columns}
\end{frame}

\begin{frame}{Training Objective: OT-CFM}
\begin{block}{Optimal Transport Conditional Flow Matching}
\begin{enumerate}
    \item Sample random flow time $t \sim U(0,1)$
    \item Sample noise $\mathbf{A}_0 \sim \mathcal{N}(0,1)$
    \item Interpolate: $\mathbf{A}_t = (1-t)\mathbf{A}_0 + t\mathbf{A}_{\text{target}}$
    \item Predict velocity: $\mathbf{v}_\theta(\mathbf{A}_t, t, \text{context})$
    \item Loss: $\mathcal{L} = \|\mathbf{v}_\theta - (\mathbf{A}_{\text{target}} - \mathbf{A}_0)\|^2$
\end{enumerate}
\end{block}

\vspace{0.5em}

\begin{block}{Inference: Generate Contact Graph}
\begin{enumerate}
    \item Start from noise: $\mathbf{A}_0 \sim \mathcal{N}(0,1)$
    \item Euler integration: $\mathbf{A}_{t+\Delta t} = \mathbf{A}_t + \mathbf{v}_\theta \cdot \Delta t$
    \item Binarize: $\mathbf{A}_{\text{final}} = \mathbf{1}[\mathbf{A}_1 > 0.5]$
    \item Symmetrize
\end{enumerate}
\end{block}
\end{frame}

% ============================================================
\section{Training Strategy}
% ============================================================

\begin{frame}{Curriculum Learning}
\begin{center}
% Figure: Curriculum learning stages
% TODO: Create figures/curriculum.svg
\fbox{\parbox{0.9\textwidth}{\centering\vspace{3.5cm}\textbf{[Figure: curriculum.svg]}\\\small Stage 1 (4-50 cells) → Stage 2 (50-200) → Stage 3 (200-500) → Stage 4 (500-1000)\vspace{1.5cm}}}
\end{center}

\begin{columns}
\column{0.24\textwidth}
\centering
\textbf{Stage 1}\\
4--50 cells\\
Simple topology\\
Fast convergence

\column{0.24\textwidth}
\centering
\textbf{Stage 2}\\
50--200 cells\\
Gastrulation\\
Cell migration

\column{0.24\textwidth}
\centering
\textbf{Stage 3}\\
200--500 cells\\
Organogenesis\\
Tissue formation

\column{0.24\textwidth}
\centering
\textbf{Stage 4}\\
500--1000 cells\\
Differentiation\\
Complex topology
\end{columns}

\vspace{0.5em}
\centering
\textbf{Progressive training on increasingly complex embryo stages}
\end{frame}

\begin{frame}{Temporal Extrapolation}
\begin{block}{The Key Challenge}
\begin{itemize}
    \item \textbf{Training data}: Early embryo (20--380 min) with ground truth contacts
    \item \textbf{Prediction target}: Late embryo (380--830 min) with NO ground truth
\end{itemize}
\end{block}

\begin{block}{Our Hypothesis}
The rules of contact formation are \textbf{learnable} and \textbf{generalizable}:
\begin{itemize}
    \item Cells with complementary adhesion molecules contact
    \item Lineage proximity predicts spatial proximity (with caveats)
    \item Morphological constraints limit possible contacts
\end{itemize}
These rules apply across developmental time.
\end{block}

\begin{block}{Uncertainty Quantification}
Generate multiple samples → Compute variance → Report confidence
\end{block}
\end{frame}

% ============================================================
\section{Validation Strategy}
% ============================================================

\begin{frame}{Validation Without Ground Truth}
\begin{columns}
\column{0.48\textwidth}
\begin{block}{1. Cross-Validation (Early Embryo)}
\begin{itemize}
    \item Leave-one-stage-out
    \item Train on stages 1,2,3 → Test on 4
    \item Metrics: AUC-ROC, Average Precision
\end{itemize}
\end{block}

\begin{block}{2. Connectome Consistency}
\begin{itemize}
    \item Adult synapses require prior contact
    \item If neurons A-B synapse in adult...
    \item ...model must predict A-B contact in embryo
    \item \textbf{Peter's Rule}: Contact is necessary for synapse
\end{itemize}
\end{block}

\column{0.48\textwidth}
\begin{block}{3. Notch Signaling Logic}
\begin{itemize}
    \item Notch requires direct contact
    \item Check: Predicted neighbors have L-R pairs?
    \item GLP-1/APX-1, LIN-12/LAG-2
    \item Known developmental inductions
\end{itemize}
\end{block}

\begin{block}{4. Cross-Species (C. briggsae)}
\begin{itemize}
    \item Conserved lineage, divergent genome
    \item Predicted patterns should be similar
    \item Evolution validates predictions
\end{itemize}
\end{block}
\end{columns}
\end{frame}

% ============================================================
\section{Implementation \& Resources}
% ============================================================

\begin{frame}{Computational Requirements}
\begin{columns}
\column{0.48\textwidth}
\begin{block}{Model Size}
\begin{tabular}{lr}
\toprule
Component & Parameters \\
\midrule
Cell Encoder & $\sim$10M \\
Pairwise Transformer & $\sim$50M \\
Flow Network & $\sim$30M \\
\midrule
\textbf{Total} & \textbf{$\sim$90M} \\
\bottomrule
\end{tabular}
\end{block}

\begin{block}{Hardware}
\begin{itemize}
    \item A100 80GB $\times$ 1--2 GPUs
    \item Training: $\sim$24 hours total
    \item Flash Attention for efficiency
\end{itemize}
\end{block}

\column{0.48\textwidth}
\begin{block}{Scalability}
\begin{tabular}{lrr}
\toprule
Stage & Cells & Memory \\
\midrule
Early & 50--200 & $\sim$4 GB \\
Mid & 200--500 & $\sim$16 GB \\
Late & 500--1000 & $\sim$48 GB \\
\bottomrule
\end{tabular}
\end{block}

\begin{block}{Inference}
\begin{itemize}
    \item 1000-cell embryo: $\sim$30s/sample
    \item 10 samples (uncertainty): $\sim$5 min
    \item Batched across time points
\end{itemize}
\end{block}
\end{columns}
\end{frame}

\begin{frame}{Current Progress}
\begin{columns}
\column{0.48\textwidth}
\begin{block}{Completed}
\begin{itemize}
    \item[\checkmark] Trimodal data integration
    \item[\checkmark] CShaper contact extraction
    \item[\checkmark] 40\% morphology coverage (94K cells)
    \item[\checkmark] Lineage proximity prior (28.5M edges)
    \item[\checkmark] GPU-accelerated pipeline
    \item[\checkmark] Unified AnnData structure
\end{itemize}
\end{block}

\column{0.48\textwidth}
\begin{block}{Next Steps}
\begin{enumerate}
    \item Implement Flow Matching Transformer
    \item scGPT embedding integration
    \item Curriculum training pipeline
    \item Validation framework
    \item Late-stage prediction
\end{enumerate}
\end{block}
\end{columns}

\vspace{0.5em}
\begin{block}{Key Metrics}
\centering
234,888 cells | 27,138 genes | 1.85M contact edges | 28.5M proximity edges
\end{block}
\end{frame}

% ============================================================
\section{Summary}
% ============================================================

\begin{frame}{The Synthesis}
\begin{center}
% Figure: Philosophy to implementation
% TODO: Create figures/synthesis.svg
\fbox{\parbox{0.85\textwidth}{\centering\vspace{3.5cm}\textbf{[Figure: synthesis.svg]}\\\small Philosophy (Organism as Context) → Architecture (Transformer) → Output (Contact Graph)\vspace{1.5cm}}}
\end{center}

\begin{block}{Key Insight}
\begin{center}
\large
The Bitter Lesson provides the \textbf{technical} justification.\\
``Organism as Context'' provides the \textbf{biological} justification.\\[0.5em]
\textbf{Transformers are not an arbitrary choice ---\\
they are the computational formalization of developmental biology.}
\end{center}
\end{block}
\end{frame}

\begin{frame}{The Question We Answer}
\begin{center}
\Large
\textit{``Given everything we know about a cell's history (lineage)\\
and current state (transcriptome),\\
can we infer its spatial relationships (contacts)\\
by understanding its place in the developing organism?''}

\vspace{2em}

\textbf{Our hypothesis: Yes.}\\[0.5em]
\textbf{Because the organism \textit{is} the context.}
\end{center}
\end{frame}

\begin{frame}
\centering
\Huge Thank You

\vspace{2em}
\Large Questions?

\vspace{2em}
\normalsize
\texttt{github.com/[repo]/NemaContext}
\end{frame}

\end{document}
